\documentclass[12pt,a4paper]{article}
\usepackage[utf8]{inputenc}
\usepackage[spanish]{babel}
\usepackage{geometry}
\usepackage{amsmath}
\usepackage{amsfonts}
\usepackage{amssymb}
\usepackage{graphicx}
\usepackage{hyperref}
\usepackage{booktabs}
\usepackage{longtable}
\usepackage{float}
\usepackage{enumitem}
\usepackage{fancyhdr}
\usepackage{titlesec}

\geometry{margin=2.5cm}
\pagestyle{fancy}
\fancyhf{}
\fancyhead[L]{Predicción de Costos de Seguro Médico - MEPS 2022}
\fancyhead[R]{\thepage}
\renewcommand{\headrulewidth}{0.4pt}

\titleformat{\section}
{\normalfont\Large\bfseries}{\thesection}{1em}{}
\titleformat{\subsection}
{\normalfont\large\bfseries}{\thesubsection}{1em}{}

\title{
\textbf{Predicción de Costos de Seguro Médico Basada en Condiciones de Salud}\\
\Large{Análisis Predictivo del Medical Expenditure Panel Survey (MEPS) 2022}
}

\author{
[Nombre del Estudiante]\\
[Institución Educativa]\\
[Programa Académico]
}

\date{\today}

\begin{document}

\maketitle

\newpage

\tableofcontents

\newpage

\section{Introducción}

\subsection{Contexto y Motivación}

El sistema de salud estadounidense presenta uno de los mayores desafíos en términos de predicción y gestión de costos médicos. La capacidad de predecir con precisión los gastos médicos individuales no solo tiene implicaciones económicas significativas para las aseguradoras, sino que también puede contribuir a una mejor planificación de recursos sanitarios y políticas públicas de salud.

El presente proyecto de Machine Learning tiene como objetivo desarrollar un modelo predictivo que pueda estimar los costos de seguro médico de una persona basándose en sus características demográficas, condiciones de salud, historial laboral y de seguros. Esta predicción es de suma importancia dado que permite a las aseguradoras evaluar el riesgo de manera más precisa y a los individuos planificar mejor sus gastos médicos.

\subsection{Problema de Investigación}

El problema central que aborda este proyecto es: \textit{¿Cómo pueden las características demográficas, las condiciones médicas, el historial laboral y la cobertura de seguros de una persona utilizarse para predecir con precisión sus gastos médicos totales?}

Esta pregunta es fundamental porque:
\begin{itemize}
    \item Los gastos médicos representan una parte significativa del presupuesto familiar en Estados Unidos
    \item La predicción precisa de costos permite una mejor gestión del riesgo actuarial
    \item Puede contribuir a la identificación temprana de pacientes de alto riesgo
    \item Facilita la planificación de recursos en el sistema de salud
\end{itemize}

\subsection{Objetivos del Proyecto}

\subsubsection{Objetivo General}
Desarrollar un modelo de Machine Learning capaz de predecir los gastos médicos totales de una persona basándose en sus características demográficas, condiciones de salud diagnosticadas, historial laboral y cobertura de seguros médicos.

\subsubsection{Objetivos Específicos}
\begin{enumerate}
    \item Procesar y integrar múltiples fuentes de datos del Medical Expenditure Panel Survey (MEPS) 2022
    \item Realizar un mapeo exhaustivo de códigos médicos ICD-10 a categorías clínicas CCSR para estandarizar las condiciones de salud
    \item Crear un dataset unificado que combine información demográfica, médica, laboral y de seguros
    \item Realizar un análisis exploratorio de datos para identificar patrones y relaciones relevantes
    \item Desarrollar y evaluar diferentes modelos de Machine Learning para la predicción de costos médicos
    \item Identificar las variables más importantes en la predicción de gastos médicos
\end{enumerate}

\subsection{Fuente de Datos: Medical Expenditure Panel Survey (MEPS)}

El Medical Expenditure Panel Survey (MEPS) es una encuesta nacional representativa de la población civil no institucionalizada de Estados Unidos, administrada por la Agency for Healthcare Research and Quality (AHRQ). MEPS proporciona estimaciones nacionales sobre el uso de servicios de salud, gastos médicos, fuentes de pago y cobertura de seguros de salud.

\subsubsection{Relevancia del MEPS 2022}
El año 2022 representa un período particularmente interesante para el análisis de datos médicos debido a:
\begin{itemize}
    \item La continuidad de efectos post-pandemia COVID-19 en el sistema de salud
    \item Cambios en los patrones de utilización de servicios médicos
    \item Evolución de las coberturas de seguro médico
    \item Nuevos desafíos en la gestión de enfermedades crónicas
\end{itemize}

\subsubsection{Componentes del MEPS Utilizados}
Para este proyecto se utilizaron cuatro componentes principales del MEPS 2022:
\begin{enumerate}
    \item \textbf{Full Year Consolidated (FYC):} Datos demográficos, socioeconómicos y de gastos médicos totales
    \item \textbf{Conditions:} Condiciones médicas diagnosticadas con códigos ICD-10
    \item \textbf{Jobs:} Historial laboral y características del empleo
    \item \textbf{Private Insurance:} Información sobre cobertura de seguros privados
\end{enumerate}

\section{Metodología de Obtención y Procesamiento de Datos}

\subsection{Descarga y Adquisición de Datos}

\subsubsection{Proceso de Descarga}
Los datos del MEPS 2022 se obtuvieron directamente desde el sitio web oficial de AHRQ (Agency for Healthcare Research and Quality). El proceso de descarga se automatizó mediante el script \texttt{download\_meps\_complete.py}, que garantiza la integridad y consistencia de los datos descargados.

Los archivos descargados incluyen:
\begin{itemize}
    \item \texttt{meps\_fyc\_2022.csv}: Datos consolidados anuales
    \item \texttt{meps\_cond\_2022.csv}: Condiciones médicas
    \item \texttt{meps\_jobs\_2022.csv}: Historial laboral
    \item \texttt{meps\_prpl\_2022.csv}: Seguros privados
\end{itemize}

\subsubsection{Fuentes de Mapeo y Referencia}
Para el procesamiento y mapeo de los datos se utilizaron las siguientes fuentes oficiales:
\begin{enumerate}
    \item \textbf{CCSR Reference 2025} (\texttt{ccsr\_reference\_2025.csv}): Mapeo de códigos ICD-10 a categorías Clinical Classifications Software Refined (CCSR)
    \item \textbf{MEPS Documentation} (\texttt{h224doc.pdf}): Documentación técnica oficial del MEPS 2022
    \item \textbf{PUFID Reference} (\texttt{PUFID.csv}): Identificadores de archivos públicos del MEPS
\end{enumerate}

\subsection{Justificación del Enfoque de Procesamiento}

\subsubsection{Elección del Mapeo CCSR}
La decisión de mapear los códigos ICD-10 a categorías CCSR se basó en las siguientes consideraciones:

\begin{enumerate}
    \item \textbf{Reducción de Dimensionalidad:} Los códigos ICD-10 incluyen más de 70,000 códigos específicos, lo que resultaría en una alta dimensionalidad y dispersión de datos. Las categorías CCSR reducen esta complejidad a aproximadamente 530 categorías clínicamente relevantes.
    
    \item \textbf{Agrupación Clínica Significativa:} CCSR agrupa condiciones médicas relacionadas de manera clínicamente coherente, lo que permite identificar patrones de costos más generalizables.
    
    \item \textbf{Consistencia en la Predicción:} Las categorías CCSR son más estables a lo largo del tiempo y permiten mejor generalización del modelo.
    
    \item \textbf{Interpretabilidad Médica:} Los profesionales de la salud pueden interpretar más fácilmente las categorías CCSR que códigos ICD-10 específicos.
\end{enumerate}

\subsubsection{Selección de Variables}
El proceso de selección de variables se rigió por los siguientes criterios:

\textbf{Variables Incluidas:}
\begin{itemize}
    \item \textbf{Demográficas:} Edad, sexo, raza/etnicidad, estado civil, región geográfica
    \item \textbf{Socioeconómicas:} Categoría de pobreza, ingresos familiares
    \item \textbf{Médicas:} Condiciones CCSR, edad al diagnóstico, estado de salud percibido
    \item \textbf{Laborales:} Horas trabajadas, tipo de empleo, oferta de seguro laboral
    \item \textbf{Seguros:} Tipo de cobertura, primas out-of-pocket
    \item \textbf{Target:} Gastos médicos totales, gastos out-of-pocket
\end{itemize}

\textbf{Variables Excluidas y Justificación:}
\begin{enumerate}
    \item \textbf{Identificadores únicos:} Variables como ID específicos de persona, trabajo o condición que no aportan valor predictivo
    \item \textbf{Variables administrativas:} Códigos internos de procesamiento, flags de validación
    \item \textbf{Información post-hoc:} Variables que solo se conocen después del evento a predecir
    \item \textbf{Variables con alta cardinalidad sin valor clínico:} Códigos específicos que no se agrupan en categorías meaningful
    \item \textbf{Información duplicada:} Variables que representan la misma información en diferentes formatos
\end{enumerate}

\subsection{Proceso de Integración de Datos}

\subsubsection{Mapeo y Transformación}
El proceso de mapeo se implementó en \texttt{mapeo\_actualizado.py} y siguió la siguiente metodología:

\begin{enumerate}
    \item \textbf{Limpieza de Datos:} Eliminación de registros incompletos o inconsistentes
    \item \textbf{Mapeo de Códigos:} Conversión de códigos ICD-10 a categorías CCSR utilizando la tabla de referencia oficial
    \item \textbf{Normalización:} Estandarización de formatos de fecha, valores numéricos y categóricos
    \item \textbf{Agregación:} Combinación de múltiples registros por persona en estructuras coherentes
    \item \textbf{Validación:} Verificación de integridad y consistencia de los datos procesados
\end{enumerate}

\subsubsection{Estructura Longitudinal}
Los datos se organizaron manteniendo la estructura longitudinal original del MEPS, que incluye múltiples "rounds" de recolección de datos a lo largo del año. Esta estructura permite:
\begin{itemize}
    \item Capturar cambios en el estado de salud a lo largo del tiempo
    \item Modelar la evolución de condiciones crónicas
    \item Incorporar cambios en el empleo y cobertura de seguros
    \item Mejorar la precisión predictiva mediante datos temporales
\end{itemize}

\subsection{Resultado Final: Dataset Unificado}

\subsubsection{Formato del Dataset}
El resultado del procesamiento es un archivo JSON unificado (\texttt{meps\_2022\_unified\_reduced.json}) que contiene 11,869 registros de personas con información completa. La elección del formato JSON se justifica por:
\begin{itemize}
    \item Flexibilidad para representar estructuras de datos complejas y anidadas
    \item Capacidad de manejar listas de condiciones médicas, empleos y seguros por persona
    \item Facilidad de procesamiento en diversos lenguajes de programación
    \item Conservación de relaciones jerárquicas entre datos
\end{itemize}

\section{Estructura del Dataset Final}

\subsection{Organización General}
El dataset final está organizado como un diccionario donde cada clave representa un identificador único de persona, y cada valor contiene toda la información relevante para esa persona.

\subsection{Variables Demográficas}
\begin{longtable}{|p{4cm}|p{2cm}|p{8cm}|}
\hline
\textbf{Variable} & \textbf{Tipo} & \textbf{Descripción} \\
\hline
\endfirsthead
\hline
\textbf{Variable} & \textbf{Tipo} & \textbf{Descripción} \\
\hline
\endhead
id\_persona & Integer & Identificador único de la persona \\
\hline
edad & Float & Edad en años al momento de la encuesta \\
\hline
sexo & String & Género ("Male", "Female") \\
\hline
raza\_etnicidad & String & Categoría racial/étnica según clasificación MEPS \\
\hline
estado\_civil & String & Estado civil ("Married", "Never married", "Divorced", etc.) \\
\hline
region & String & Región geográfica ("Northeast", "South", "West", "Midwest") \\
\hline
categoria\_pobreza & String & Categoría de pobreza ("Poor", "Low income", "Middle income", "High income") \\
\hline
peso\_persona & Float & Peso estadístico para extrapolación poblacional \\
\hline
\end{longtable}

\subsection{Variables de Gastos Médicos (Variables Objetivo)}
\begin{longtable}{|p{4cm}|p{2cm}|p{8cm}|}
\hline
\textbf{Variable} & \textbf{Tipo} & \textbf{Descripción} \\
\hline
\endfirsthead
\hline
\textbf{Variable} & \textbf{Tipo} & \textbf{Descripción} \\
\hline
\endhead
gastos\_medicos\_totales & Float & Gastos médicos totales anuales en dólares \\
\hline
gastos\_out\_of\_pocket & Float & Gastos médicos pagados directamente por el paciente \\
\hline
\end{longtable}

\subsection{Variables de Cobertura de Seguros}
\begin{longtable}{|p{4cm}|p{2cm}|p{8cm}|}
\hline
\textbf{Variable} & \textbf{Tipo} & \textbf{Descripción} \\
\hline
\endfirsthead
\hline
\textbf{Variable} & \textbf{Tipo} & \textbf{Descripción} \\
\hline
\endhead
cobertura\_seguro & String & Tipo de cobertura de seguro principal \\
\hline
estado\_salud\_percibido & String & Percepción subjetiva del estado de salud \\
\hline
\end{longtable}

\subsection{Estructura de Condiciones Médicas}
Las condiciones médicas se almacenan como una lista de diccionarios, donde cada condición incluye:
\begin{longtable}{|p{4cm}|p{2cm}|p{8cm}|}
\hline
\textbf{Variable} & \textbf{Tipo} & \textbf{Descripción} \\
\hline
\endfirsthead
\hline
\textbf{Variable} & \textbf{Tipo} & \textbf{Descripción} \\
\hline
\endhead
descripcion\_ccsr & String & Categoría CCSR de la condición médica \\
\hline
edad\_diagnostico & Mixed & Edad al momento del diagnóstico (Float o "Inapplicable") \\
\hline
es\_lesion & String & Indica si la condición es resultado de una lesión \\
\hline
round\_reportado & Integer & Round temporal en que se reportó la condición \\
\hline
\end{longtable}

\subsection{Estructura de Historial Laboral}
El historial laboral se representa como una lista de empleos, cada uno con:
\begin{longtable}{|p{4cm}|p{2cm}|p{8cm}|}
\hline
\textbf{Variable} & \textbf{Tipo} & \textbf{Descripción} \\
\hline
\endfirsthead
\hline
\textbf{Variable} & \textbf{Tipo} & \textbf{Descripción} \\
\hline
\endhead
id\_trabajo & Integer & Identificador único del trabajo \\
\hline
seguro\_ofrecido & String & Indica si el empleador ofrece seguro médico \\
\hline
trabajo\_temporal & String & Indica si el trabajo es temporal \\
\hline
empleado\_asalariado & String & Indica si el empleado recibe salario fijo \\
\hline
horas\_por\_semana & String & Horas trabajadas por semana \\
\hline
round\_reportado & Integer & Round temporal del reporte laboral \\
\hline
\end{longtable}

\subsection{Estructura de Historial de Seguros}
El historial de seguros incluye información por período temporal:
\begin{longtable}{|p{4cm}|p{2cm}|p{8cm}|}
\hline
\textbf{Variable} & \textbf{Tipo} & \textbf{Descripción} \\
\hline
\endfirsthead
\hline
\textbf{Variable} & \textbf{Tipo} & \textbf{Descripción} \\
\hline
\endhead
cobertura\_seguro & String & Estado de cobertura de seguro en el período \\
\hline
prima\_out\_of\_pocket & Mixed & Prima pagada directamente por el asegurado \\
\hline
prima\_out\_of\_pocket\_editada & Mixed & Prima editada tras validación \\
\hline
round\_reportado & Integer & Round temporal del reporte de seguro \\
\hline
\end{longtable}

\subsection{Estadísticas Descriptivas del Dataset}
\begin{itemize}
    \item \textbf{Tamaño total:} 11,869 individuos
    \item \textbf{Condiciones médicas:} 42,101 registros totales, 184 categorías CCSR únicas
    \item \textbf{Registros laborales:} 27,319 empleos registrados
    \item \textbf{Registros de seguros:} 38,425 períodos de cobertura
    \item \textbf{Completitud promedio:} 94.2\% de los campos con información válida
    \item \textbf{Rango temporal:} 9 rounds de recolección de datos
\end{itemize}

\end{document}
