\documentclass[12pt,a4paper]{article}
\usepackage[utf8]{inputenc}
\usepackage[spanish]{babel}
\usepackage{geometry}
\usepackage{amsmath}
\usepackage{amsfonts}
\usepackage{amssymb}
\usepackage{graphicx}
\usepackage{hyperref}
\usepackage{booktabs}
\usepackage{longtable}
\usepackage{float}
\usepackage{enumitem}
\usepackage{fancyhdr}
\usepackage{titlesec}

\geometry{margin=2.5cm}
\pagestyle{fancy}
\fancyhf{}
\fancyhead[L]{Predicción de Costos de Seguro Médico - MEPS 2022}
\fancyhead[R]{\thepage}
\renewcommand{\headrulewidth}{0.4pt}

\titleformat{\section}
{\normalfont\Large\bfseries}{\thesection}{1em}{}
\titleformat{\subsection}
{\normalfont\large\bfseries}{\thesubsection}{1em}{}

\title{Predicción personalizada de primas de seguro médico usando Machine Learning}
\author{Diego Puentes, Universidad de La Habana}
\date{Julio 2025}

\begin{document}

\maketitle

\section*{Descripción del problema}
El objetivo de este trabajo es predecir, a partir de características personales y de salud de los individuos, el coste de la prima ``out of pocket'' de su seguro médico utilizando técnicas de aprendizaje automático. Se busca estimar, para cada persona, el rango de valores que sería razonable pagar por su seguro, en función de lo que pagan personas con características similares.

\section*{Origen y descripción de los datos}
Los datos utilizados provienen de tres fuentes principales: el Medical Expenditure Panel Survey (MEPS), el Clinical Classifications Software Refined (CCSR) y el Crosswalk for Clinical Information Reporting (CCIR).

\begin{itemize}
    \item \textbf{MEPS}: Encuesta nacional de gastos médicos en Estados Unidos, que recopila información detallada sobre el uso de servicios de salud, gastos y seguros médicos de la población.
    \item \textbf{CCSR}: Herramienta de clasificación que agrupa diagnósticos clínicos de acuerdo a códigos ICD-10, facilitando el análisis de condiciones de salud.
    \item \textbf{CCIR}: Tabla de correspondencia que permite mapear códigos y categorías clínicas entre diferentes sistemas de clasificación.
\end{itemize}

Los datos originales se encontraban en archivos CSV ubicados en la carpeta \texttt{raw}. Entre las columnas presentes en estos archivos se incluyen identificadores de persona, edad, sexo, raza/etnia, estado civil, región, condiciones de salud codificadas, y variables relacionadas con el seguro y los gastos médicos.

\section*{Procesamiento y mapeo de los datos}
Para mejorar la interpretabilidad, los nombres de las variables fueron mapeados a descripciones más explícitas utilizando los archivos de usuario SAS provistos por MEPS (archivos .txt). Posteriormente, se construyó un archivo JSON unificado (\texttt{meps2022\_unified\_reduced.json}) que integra la información relevante de los distintos archivos procesados.

La estructura del JSON unificado contiene, para cada individuo, las variables personales, demográficas y de salud, así como la información de gastos y seguro. Algunas variables originales fueron omitidas en la construcción del JSON final por carecer de relevancia directa para la predicción de la prima o por contener información redundante o incompleta. Esta selección se realizó para optimizar la calidad y utilidad del dataset final para el modelado predictivo.

\end{document}
