\documentclass[12pt,a4paper]{article}
\usepackage[utf8]{inputenc}
\usepackage[spanish]{babel}
\usepackage{geometry}
\usepackage{amsmath}
\usepackage{amsfonts}
\usepackage{amssymb}
\usepackage{graphicx}
\usepackage{hyperref}
\usepackage{booktabs}
\usepackage{longtable}
\usepackage{float}
\usepackage{enumitem}
\usepackage{fancyhdr}
\usepackage{titlesec}

\geometry{margin=2.5cm}
\pagestyle{fancy}
\fancyhf{}
\fancyhead[L]{Predicción de Costos de Seguro Médico - MEPS 2022}
\fancyhead[R]{\thepage}
\renewcommand{\headrulewidth}{0.4pt}

\titleformat{\section}
{\normalfont\Large\bfseries}{\thesection}{1em}{}
\titleformat{\subsection}
{\normalfont\large\bfseries}{\thesubsection}{1em}{}

\title{Predicción personalizada de primas de seguro médico usando Machine Learning}
\author{Diego Puentes, Universidad de La Habana}
\date{Julio 2025}

\begin{document}

\maketitle

\section*{Descripción del problema}
El objetivo de este trabajo es predecir, a partir de características personales y de salud de los individuos, el coste de la prima ``out of pocket'' de su seguro médico utilizando técnicas de aprendizaje automático.

\section*{Origen y descripción de los datos}
Los datos utilizados provienen de tres fuentes principales: el Medical Expenditure Panel Survey (MEPS), el Clinical Classifications Software Refined (CCSR) y el Crosswalk for Clinical Information Reporting (CCIR).

\begin{itemize}
    \item \textbf{MEPS}: Encuesta nacional de gastos médicos en Estados Unidos, que recopila información detallada sobre el uso de servicios de salud, gastos y seguros médicos de la población.
    \item \textbf{CCSR}: Herramienta de clasificación que agrupa diagnósticos clínicos de acuerdo a códigos ICD-10, facilitando el análisis de condiciones de salud.
    \item \textbf{CCIR}: Tabla de correspondencia que permite mapear códigos y categorías clínicas entre diferentes sistemas de clasificación.
\end{itemize}




\section*{Primer procesamiento y mapeo de los datos}

Se implementó una función de mapeo para convertir las columnas de los archivos CSV originales en nombres más entendibles y descriptivos, utilizando los archivos de usuario SAS provistos por MEPS (archivos .txt). Además, se realizó un análisis exploratorio de los datos, cuyos resultados principales se resumen en la siguiente tabla:

\begin{table}[H]
\centering
\begin{tabular}{ll}
\toprule
\textbf{Indicador} & \textbf{Valor} \\
\midrule
\multicolumn{2}{l}{\textbf{Demografía (fyc)}} \\
Columnas disponibles & dwelling\_unit\_id, person\_id, person\_unique\_id, panel\_number, age\_last\_birthday, sex, race\_ethnicity, \\ 
 & marital\_status\_2022, region\_2022, total\_healthcare\_exp\_2022, total\_out\_of\_pocket\_exp\_2022, \\ 
 & poverty\_category\_2022, insurance\_coverage\_2022, perceived\_health\_status, person\_weight\_2022 \\
Edad (mín, máx, Q1, Q3, media, mediana) & 0.0, 85.0, 23.0, 64.0, 43.56, 45.0 \\
\midrule
Cantidad de personas por raza &  \\
\quad Non-Hispanic White only & 12211 \\
\quad Hispanic & 4883 \\
\quad Non-Hispanic Black only & 3244 \\
\quad Non-Hispanic Asian only & 1220 \\
\quad Non-Hispanic Other/multi-race & 873 \\
\midrule
Cantidad de personas por estado civil &  \\
\quad Married & 8602 \\
\quad Never married & 5495 \\
\quad Under 16 - not applicable & 3765 \\
\quad Divorced & 2546 \\
\quad Widowed & 1619 \\
\quad Separated & 397 \\
\quad -7 & 6 \\
\quad -8 & 1 \\
\midrule
Cantidad de personas por región &  \\
\quad South & 8602 \\
\quad West & 5693 \\
\quad Midwest & 4498 \\
\quad Northeast & 3443 \\
\quad Inapplicable & 195 \\
\midrule
Cantidad de personas por categoría de pobreza &  \\
\quad High income & 8282 \\
\quad Middle income & 6269 \\
\quad Poor/negative & 3725 \\
\quad Low income & 3105 \\
\quad Near poor & 1050 \\
\midrule
Cantidad de personas individuales & 22431 \\
\midrule
\multicolumn{2}{l}{\textbf{Condiciones médicas (cond)}} \\
Columnas disponibles & person\_unique\_id, condition\_id, panel\_number, condition\_round, age\_at\_diagnosis, injury\_flag, icd10\_code, ccsr\_category\_1 \\
Condiciones médicas distintas & 206 \\
Media de condiciones por persona & 4.80 \\
Top 5 condiciones más comunes &  \\
\quad CIR007 & 5391 \\
\quad END010 & 4268 \\
\quad MUS010 & 3061 \\
\quad END002 & 2334 \\
\quad MBD005 & 2158 \\
\midrule
\multicolumn{2}{l}{\textbf{Primas y pagos (prpl)}} \\
Estadísticas out\_of\_pocket\_premium\_edited & mín: 0.0, máx: 4583.33, media: 306.68, mediana: 212.5, Q1: 70.0, Q3: 433.33 \\
Cantidad de valores válidos & 13075 \\
\midrule
\multicolumn{2}{l}{\textbf{Empleo (jobs)}} \\
Estadísticas hours\_per\_week & mín: 1.0, máx: 168.0, media: 35.83, mediana: 40.0, Q1: 30.0, Q3: 40.0, válidos: 36126 \\
Estadísticas hourly\_wage & mín: 0.0, máx: 115.0, media: 20.65, mediana: 18.0, Q1: 15.0, Q3: 24.0, válidos: 13522 \\
\bottomrule
\end{tabular}

\caption{Resumen de los principales resultados del análisis exploratorio de los datos procesados.}
\end{table}

\vspace{1em}
\noindent
\textbf{Columnas disponibles en los datasets:}

\begin{itemize}
    \item \textbf{Condiciones médicas (cond):} person\_unique\_id, condition\_id, panel\_number, condition\_round, age\_at\_diagnosis, injury\_flag, icd10\_code, ccsr\_category\_1
    \item \textbf{Características personales y demográficas (fyc):} dwelling\_unit\_id, person\_id, person\_unique\_id, panel\_number, age\_last\_birthday, sex, race\_ethnicity, marital\_status\_2022, region\_2022, total\_healthcare\_exp\_2022, total\_out\_of\_pocket\_exp\_2022, poverty\_category\_2022, insurance\_coverage\_2022, perceived\_health\_status, person\_weight\_2022
    \item \textbf{Empleo (jobs):} person\_unique\_id, job\_id, panel\_number, round\_number, insurance\_offered, temporary\_job, salaried\_employee, hourly\_wage, hours\_per\_week
    \item \textbf{Historial de seguros (prpl):} person\_unique\_id, panel\_number, round\_number, insurance\_coverage, out\_of\_pocket\_premium, out\_of\_pocket\_premium\_edited
\end{itemize}

No todas estas columnas ofrecían información relevante para el objetivo del trabajo. Para facilitar el análisis y la integración de la información, se decidió crear un archivo JSON unificado por persona, con la siguiente estructura anidada de campos principales:

\begin{verbatim}
{
  "edad": ,
  "sexo": ,
  "raza_etnicidad": ,
  "estado_civil": ,
  "region": ,
  "categoria_pobreza": ,
  "cobertura_seguro": ,
  "estado_salud_percibido": ,
  "condiciones_medicas_actuales": [
    {
      "descripcion_ccsr": ,
      "edad_diagnostico": 
    },
    ...
  ],
  "condiciones_medicas_pasadas": [],
  "historial_empleo": [
    {
      "seguro_ofrecido": ,
      "trabajo_temporal": ,
      "empleado_asalariado": ,
      "salario_por_hora": ,
      "horas_por_semana": 
    },
    ...
  ],
  "historial_seguros": [
    {
      "cobertura_seguro": ,
      "prima_out_of_pocket_editada": 
    },
    ...
  ]
}
\end{verbatim}

\end{document}
